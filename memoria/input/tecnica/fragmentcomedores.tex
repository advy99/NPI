\section{Fragment Comedores, Calendario y Catálogo Biblioteca}


El objetivo de estos fragment es facilitar al usuario que pueda ver el contenido de diferentes docuemntos web y paginas que le proporcionaran información de utilidad

El fragment Comedores mostrará el menú semanal que ofrece la UGR en todos sus comedores.

El fragment Calendario mostrará el calendario academico de la UGR.

El fragment Catálogo Biblioteca simplemente muestra la pagina web de la biblioteca de la UGR que nos permite consultar su catálogo de libros y obtener información útil de los mismos.

Para ello vamos a hacer uso principalmente de una \textbf{WebView} que va a cargar el pdf \\ \textbf{http://scu.ugr.es/?theme=pdf} apoyándonos en Google Drive o la pagina web necesaria.

También vamos a hacer uso de una \textbf{ProgressBar} que nos indicará como va el proceso de carga de la url. Vemos necesario usarla porque si no parece que la pantalla no está cargando y el fragment no funciona bien.

A continuación vamos a ver las variables y los métodos que forman estos fragments, y que hace cada uno.

\subsection{Variables}

\begin{itemize}

\item{lateinit var webview: WebView}
\item{lateinit var barra: ProgressBar}
\item{lateinit var pdf: String}

\end{itemize}

Las dos primeras variables son la propia \textbf{WebView} y la \textbf{ProgressBar} de las que he hablado antes. Ambas serán inicializadas en el método \textbf{OnCreateView} usando los id del layout.
La tercera variable corresponde al string que guardará la url del contenido a mostrar y que pasaremos como parámetro al método \textbf{loadUrl} de la WebView.

\newpage

\subsection{Métodos}

Los métodos que voy a explicar a continuación son:

\begin{itemize}
\item{onCreateView}
\end{itemize}

\subsubsection{Método OnCreateView}

Este método comienza inicializando las variables \textbf{webview} y \textbf{barra} explicadas anteriormente mediante \textbf{findViewById} y pasándole la Id del layout.

También le asignamos a \textbf{pdf} la url del pdf que queramos mostrar. Esto no es necesario para el fragment catálogo ya que no muestra la información a través de un pdf.

Una vez hecho esto, vamos a configurar la \textbf{WebView} de tal forma que pueda usar JavaScript

\begin{lstlisting}[language=Kotlin]
webview.settings.javaScriptEnabled = true
\end{lstlisting}

y que el usuario pueda usar gestos para acercar y alejar la WebView a su antojo

\begin{lstlisting}[language=Kotlin]
webview.settings.setSupportZoom(true)
webview.settings.builtInZoomControls = true
\end{lstlisting}

También haremos invisible la webview nada más iniciar el fragment para que lo que se vea sea la \textbf{ProgressBar} cargando.

Ahora lo que nos quedaría sería modificar dos objetos internos de nuestra \textbf{WebView}. El primero sería su objeto \textbf{WebChromeClient()}. Este objeto tiene un método llamado \textbf{onProgressChanged} que nos permite ver cómo va el progreso de carga de la WebView, es por esto que lo usaremos para ocultar la webview si el progreso es menor a 100, al mismo tiempo que vamos mostrando como avanza la progressBar. Cuando el progreso llegue a 100, mostraremos la WebView y ocultaremos la ProgressBar.

El otro objeto es el \textbf{WebViewClient()}, el cual tiene un método llamado \textbf{shouldOverrideUrlLoading} y que tiene como parámetro una \textbf{request} que le pasaremos a \textbf{webView.loadUrl(request)}. De esta forma, cada vez que hagamos \textit{"tap"} sobre algún enlace, se recargará la WebView mostrando el enlace.

Finalmente nos quedaría mostrar la URL del contenido correspondiente  mediante el método \textbf{loadUrl} de nuestra WebView.

\begin{figure}[H]
  \begin{subfigure}{0.5\textwidth}
    \centering
    % include first image
    \includegraphics[width=1\linewidth]{progressBar.jpeg}
    \caption{Barra de progreso}
    \label{fig:sub-first}
  \end{subfigure}
  \begin{subfigure}{0.5\textwidth}
    \centering
    % include second image
    \includegraphics[width=1\linewidth]{comedores.jpeg}
    \caption{PDF comedores cargado}
    \label{fig:sub-second}
  \end{subfigure}
\end{figure}

\begin{figure}[H]
  \begin{subfigure}{0.5\textwidth}
    \centering
    % include first image
    \includegraphics[width=1\linewidth]{calendario.jpg}
    \caption{Calendario}
    \label{fig:sub-first}
  \end{subfigure}
  \begin{subfigure}{0.5\textwidth}
    \centering
    % include second image
    \includegraphics[width=1\linewidth]{catalogo.jpg}
    \caption{Catálogo Biblioteca}
    \label{fig:sub-second}
  \end{subfigure}
\end{figure}

