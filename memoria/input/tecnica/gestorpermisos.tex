\section{Gestor Permisos}
Utilizamos un objeto en kotlin llamado GestorPermisos con el fin de poder centralizar todos aquellos métodos que necesiten solicitar permisos al usuarios. Los objetos en kotlin son el equivalente a las clases singleton en Java, es decir, objetos que solo pueden instanciarse una sola vez.

\subsection{Variables}
Contamos con dos únicas variables:

\begin{itemize}
	\item mLocationPermissionGranted
	\item LOCATION\_PERMISSION\_REQUEST\_CODE
\end{itemize}

La variable \textbf{mLocationPermissionGranted} simplemente funciona como un flag que indica si los permisos han sido otorgados por el usuario o no. La variable \textbf{LOCATION\_PERMISSION\_REQUEST\_CODE} simplemente es el código que le damos a la solicitud de los permisos de ubicación.

\subsection{Métodos}

Este gestor de permisos posee los siguientes métodos:

\begin{itemize}
	\item locationPermissionGranted
	\item getLocationPermission
	\item onRequestPermissionResult
\end{itemize}


\subsubsection{locationPermissionGranted}
Este método simplemente funciona como consultor del valor de la variable \textbf{mLocationPermissionGranted}.

\subsubsection{llocationPermission}
Este es el método que se encarga de solicitar los permisos. Simplemente comprueba uno a uno que los permisos necesarios hayan sido obtenidos correctamente y actualiza el valor de \textbf{mLocationPermissionGranted}.

\subsubsection{onRequestPermissionResult}
Este método se encarga de comprobar que los permisos han sido otorgados y actualizar \textbf{mLocationPermissionGranted} en caso de que no sea así.


