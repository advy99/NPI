\section{Paradigmas de interacción}

Nuestra aplicación implementará una navegación y funcionalidades utilizando tanto gestos como voz. Además se plantea la posibilidad de que la aplicación sirva de un apoyo para guiarte por granada mediante el uso de las google glass, o la opción de obtener mediante Wear OS el estado del usuario y permitir a la aplicación actuar conforme a esta información.

\subsection{Integración de gestos}
A parte de implementar los gestos básicos presentes en cada aplicación (desplazar el dedo, doble toque, etc$\ldots$). Se desea implementar diversos gestos para poder acceder fácilmente desde cualquier parte de la aplicación, por ejemplo, al dibujar una C se muestra el menú de comedores de la semana.

\subsection{Integración Google Glass}

Nuestra aplicación será capaz de conectarse con el dispositivo inteligente de Google, permitiendo que mientras el usuario camina por la calle se le de información sobre los lugares que visita, así como instrucciones para moverse por la ciudad de cara a que el usuario la conozca.

\subsection{Integración con SmartWatchs}

También será posible conectar la aplicación con relojes inteligentes, de cara a obtener información biometrica del usuario si este lo desea con el objetivo de orientar al usuario de su estado mientras utiliza la aplicación, por ejemplo si se trata de un estudiante de camino a su facultad y tiene el pulso demasiado alto, mostrar una advertencia para que baje el ritmo por su seguridad.

\subsection{Integración de voz}

Arandita es un bot con el que puedes chatear y hablar, sobre la UGR y sus distintos centros, además de ofrecerte interesantes curiosidades sobre la universidad.

También se quiere implementar la opción de hablar con el bot para solicitar  información sobre los horarios de las facultades, su calendario académico, menú de comedores…

Arandita también permite la grabación de recordatorios para guardar fechas de exámenes, entregas u otros eventos importantes para el estudiante.

Se desea implementar una opción que permita a Arandita describa lo que aparezca en pantalla para los usuarios con discapacidad visual.
